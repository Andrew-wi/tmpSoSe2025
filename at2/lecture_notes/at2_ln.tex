\documentclass{article}

\usepackage{technical_notes_1_20250424}

\makeindex

\begin{document}

\tableofcontents

\section{Lecture 1}
\label{sec:lecture1}

Today, we reviewed:
\begin{enumerate}
\item Homology.
\item How to construct cohomology.
\item Made a statement about a sequence splitting.
\item The sequence contains $\ker \delta_n$, which we don't understand.
\end{enumerate}

Understanding this $\ker \delta_n$ will be the topic of the next lecture. For more details, see my handwritten notes.

Recall that \textbf{homology}\index{homology} is represented as taking a space $X$, making a chain complex $C_i$ out of it, and then by a purely algebraic manipulation creating a homology $H_i$. For a cochain complex, we dualize $C_i$, then take a cohomology $H^i$.

\begin{equation}
\begin{tikzcd}[column sep=5.0em]
  \text{Space } X \arrow[r, rightsquigarrow] & \text{chain complex } C_i \arrow[r, rightsquigarrow, "\text{purely algebraic}"] \arrow[d, "\text{dualize}", rightsquigarrow] & \text{homology} H_i \\
  & \text{cochain complex } C_i \ar[r, rightsquigarrow] & \text{cohomology } H^{i}
\end{tikzcd}
\end{equation}

Suppose now that we have a chain complex $C_{i}$ of free Abelian groups:
\begin{equation}
  \dots C_{n + 1} \xrightarrow{\partial} C_{n} \rightarrow C_{n - 1} \rightarrow \dots.
\end{equation}
Take an abelian group $G$ (feel free to think about $G = \Z$) and consider the dual
\begin{equation}
  (C_n)^{*} = \text{Hom}(C_n, G) \text{, an additive group.}
\end{equation}

We have no natural map from $C_n^{*} \rightarrow C_{n - 1}^{*}$, but we do have one from 
\begin{align}
\label{eq:1}
  C_n^{*} &\rightarrow C_{n + 1}^{*} \\
  \partial^{*} : \text{Hom}(C_n, G) &\rightarrow \text{Hom}(C_{n+1}, G) \\
  \varphi &\mapsto \varphi \circ \partial
\end{align}

giving us 
\begin{align}
\label{eq:2}
  \dots \rightarrow C_{n - 1}^{*} \rightarrow C_n^{*} \xrightarrow{\delta} C_{n + 1}^{*} \rightarrow C_{n+2}^{*} \rightarrow \dots
\end{align}

Note that we use the sign convention
\begin{equation}
  \delta : C_n^{*} \rightarrow C_{n+1}^{*} \text{ in } (-1)^{n + 1}\partial^{*}
\end{equation}
which is \textbf{\textit{not}} used in Hatcher \& Bredon.

\begin{Definition}{Cochain Complex}
. A \textbf{cochain complex}\index{cochain complex} $C^{\bullet}$ is a collection of Abelian groups $C^n$ with codifferentials $\delta_n$ such that $\delta_{n + 1}\delta_n = 0$.
\end{Definition}

Thus,
\begin{equation}
  \mathcal{H}om(C_{\bullet}, G) = \text{Hom}(C_i, G) \text{ with differential } \delta = (-1)^{n + 1}\partial^{*}
\end{equation}
is a cochain complex, as defined above. Be cautious of $\text{Hom}$ vs. $\mathcal{H}om$.

\begin{Definition}{Cohomology}
. For a cochain complex $C^{\bullet}$, define \textbf{cohomology}\index{cohomology} as
\begin{equation}
  H^i(C^{\bullet}) = \frac{\ker \delta_i}{\text{im} \delta_{i-1}}.
\end{equation}
\end{Definition}

If $C_{\bullet}$ is a chain complex, then any choice of abelian group $G$ gives a cochain complex
\begin{equation}
  \mathcal{H}om(C_{\bullet}, G),
\end{equation}
hence a cohomology and thus gohomology groups, denoted as $H^{*}(C_{\bullet}, G)$.

If $C_{\bullet}$ is the singular cochain complex of a space $X$, then put
\begin{equation}
  H^{*}(X', G) = H^{*}(\mathcal{H}om(C, G)).
\end{equation}

\begin{Example}{Singuluar Cochain Complex}
.  Let $C_{\bullet}$ be:
\begin{equation}
  \begin{tikzcd}
    \dots  \ar[r] & 0 \ar[r] & \Z \ar[r, "0"] & \Z \ar[r, "\cdot 2"] & \Z \ar[r, "0"] & \Z \ar[r] & 0.
  \end{tikzcd}
\end{equation}

Then we have:
\begin{equation}
  H_{*}(C_{\bullet}) =
  \begin{cases}
    \Z & \quad n = 0,\\
    \Z / 2\Z & \quad n = 1,\\
    0 & \quad n = 2,\\
    \Z & \quad n = 3.
  \end{cases}
\end{equation}
If we let $G = \Z$, and dualize, then we obtain:
\begin{equation}
  \begin{tikzcd}
    \dots & \ar[l] 0 & \ar[l] \Z & \ar[l, "0"] \Z & \ar[l, "\cdot 2"] \Z & \ar[l, "0"] \text{\normalfont Hom}(\Z, \Z) \cong \Z & \ar[l] 0.
  \end{tikzcd}
\end{equation}
has cohomology:
\begin{equation}
  H^{*}(C_{\bullet}) =
  \begin{cases}
    \Z & \quad n = 0,\\
    0 & \quad n = 1,\\
    \Z / 2\Z & \quad n = 2,\\
    \Z & \quad n = 3.
  \end{cases}
\end{equation}
These are \textbf{not} the duals of the homology $H_{*}$. Moreover, $G = \Z/2\Z$ has different results. 
\end{Example}

\textbf{\textit{Note: Functoriality.}} If $C_{\bullet} \rightarrow D_{\bullet}$ is a map of cochain complexes, dualizing gives
\begin{align}
  \mathcal{H}om (D_{\bullet}, G) &\rightarrow \mathcal{H}(C_{\bullet}, G) \\
  \varphi &\mapsto \varphi \circ f
\end{align}
which induces a map
\begin{equation}
  H^{*}(\mathcal{H}om(D_{\bullet}, G)) \rightarrow H^{*}(\mathcal{H}om(C_{\bullet}, G)).
\end{equation}

We want to now relate the cohomology of $\mathcal{H}om(C_{\bullet}, G)$ to the homology of $C_{\bullet}$.

\begin{Lemma}{Cohomology and Homology}
. For any chain complex of free Abelian groups, and any Abelian group $G$, there is a natural surjection:
\begin{equation}
  h : H^n(C_{\bullet}; G) \rightarrow \text{\normalfont Hom}(H_n(C_\bullet; G))
\end{equation}
and the sequence
\begin{equation}
  \begin{tikzcd}
    0 \ar[r] & \ker(h) \ar[r] & H^n(C_{\bullet}; G) \ar[r] & \ar[l, dashed, red, bend right=30] \text{\normalfont Hom}(H_{n}(C_{\bullet}; G)) \ar[r] & 0
  \end{tikzcd}
\end{equation}
\coloruwave{red}{splits}.
\end{Lemma}

\begin{proof}
  The proof can be found in my live notes. Please review it in detail.
\end{proof}

\printindex

\end{document}
