\documentclass{article}

\usepackage{technical_notes_1_20250424}

\makeindex

\begin{document}

\tableofcontents

\section{Lecture 1}
\label{sec:lecture1}

Overview of the math part of the lecture. FV = Friedli, Velenik.

\begin{enumerate}
\item Ising model: existence of thermodynamic limit of pressure/free energy. We will go over Peierl's argument and phase transitions. \textit{Book references:} FV Chapter 3.
\item Gibbs measures in infinite volume. DLR conditions: Debrushin, Lanford, Ruelle. \textit{Book references:} FV Chapter 6.
\item Mermin-Wagner theorem, absence of continuous symmetry breaking in $d = 1, 2$. \textit{Book references:} FV Chapter 9.
\item If time admits: reflection positivity \& existence of symmetry breaking in $d = 3$. \textit{Book references:} FV Chapter 10.
\end{enumerate}

\subsection{The Ising Model}
\label{sec:ising_model}

\subsubsection{The Model}

\textbf{\textit{Notation:}} we will denote by~\index{$\Lambda$}~$\Lambda \Subset \Z^d$ that $\Lambda \subset \Z^d$, and that $\Lambda$ is finite, and non-empty. Often, it will be the case that $\Lambda = \{ 1 , \dots , L \}^d, \; L \in \N$; i.e., that $\Lambda$ consists of a square grid (a ``lattice'').

\textbf{Configuration space}\index{configuration space}. We denote a configuration space by $\Omega_{\Lambda} := \{ -1, \, +1 \}^{\Lambda}$. Here,
\begin{equation}
  \omega \in \Omega_{\Lambda}, \; \omega = (\omega_i)_{i \in\Lambda}, \; \omega_i\in \{ \pm 1\}.
\end{equation}
More verbosely:~\index{$\Omega_{\Lambda}$}~$\Omega_{\Lambda}$ is the set of functions assigning either $+1$ or $-1$ onto each vertex of the lattice, and each $\omega$ is an individual ``configuration'' which is a collection of $\{ \pm 1 \}$ assigned to each vertex $i \in \Lambda$.

We also let $h \in \R$ be an external magnetic field. Now, we are in place to define the ``\textbf{Ising Hamiltonian}''\index{Ising Hamiltonian}, denoted by $\sH_{\Lambda; h}$:
\begin{align*}
  \mathscr{H}_{\Lambda; h} : \Omega_{\Lambda} &\rightarrow \R, \\
  \omega = (\omega_i)_{i \in \Lambda} &\mapsto \sH_{\Lambda;h}(\omega) = {-} \sum\limits_{\substack{\{i, j\} \subset \Lambda \\ i, j \text{ n.n.}}} \omega_i \omega_j - h \sum\limits_{i \in \Lambda} \omega_i
\end{align*}
where ``n.n.'' means ``nearest neighbors in $\Z^d$''.

A natural question that arises is: what is (are) the minimizer(s) of $\sH$?
\begin{align*}
  h > 0 &: \omega_i = 1 \; \forall i,\\
  h < 0 &: \omega_i = -1 \; \forall i, \\
  h = 0 &: 2 \text{ minimizers, all } +1 \text{ or all } -1.
\end{align*}

We also introduce now the \textbf{partition function}\index{partition function} (Zustandssamme):
\begin{eqnarray}
\label{eq:2}
Z_{\Lambda} = \sum\limits_{\omega\in\Omega_{\Lambda}} e^{-\beta\sH_{\Lambda;h}(\omega)}.
\end{eqnarray}
where $\beta = \frac{1}{T}$ is the \textbf{inverse temperature}\index{inverse temperature}.

With this, we may then define the \textbf{Gibbs measure}\index{Gibbs measure}, which is a probability measure on $\Omega_{\Lambda}$:
\begin{eqnarray}
\label{eq:1}
\mu_{\lambda;\beta,h}(\{\omega\}) = \frac{1}{Z_{\Lambda}(\beta,h)} e^{-\beta\sH_{\lambda;h}(\omega)}
\end{eqnarray}
Let's take note of a couple important regimes in different values of the parameter $\beta$. First, when $\beta = 0$, we get a uniform distribution: it's completely flat. When $\beta \to \infty$, then the measure concentrates on the minimzer(s).

Now, we can define \textbf{pressure/free energy}\index{pressure}\index{free energy}:
\begin{eqnarray}
\label{eq:3}
  \psi_{\Lambda}(\beta,h) := \frac{1}{\beta|\Lambda|}\log Z_{\Lambda} (\beta, h).
\end{eqnarray}
This quantity is dependent upon the system size, inverse temperature, and magnetic field, as we might intuitively expect.

We can also define the \textbf{total magnetization}\index{total magnetization}, which is the map:
\begin{align}
  M_{\lambda} : \Omega_{\Lambda} &\to \R, \\
  \omega &\mapsto M_{\Lambda}(\omega) = \sum\limits_{i \in \Lambda} \omega_i.
\end{align}
Physically, this is just the sum of the spins on each vertex of the lattice $\Lambda$.\\

\noindent\textbf{Observation.} We now make an observation regarding the dependence of $\psi_{\Lambda}$ on $h$:
\begin{align}
\label{eq:6}
  \frac{\partial}{\partial h} \psi_{\Lambda}(\beta,h) &= \frac{1}{|\Lambda|}\sum\limits_{\omega\in\Omega_{\Lambda}} M_{\Lambda}(\omega) e^{-\beta\sH_{\Lambda,h}(\omega)} \frac{1}{Z_{\Lambda}(\beta,h)} \\
                                                      &= \frac{1}{|\Lambda|} \langle M_{\Lambda} \rangle_{\Lambda, \beta, h} \\
  &=: m_{\Lambda}(\beta, h),
\end{align}
where $m_{\Lambda}(\beta, h)$ is the \textbf{average magnetization}\index{average magnetization} per unit volume.

\section{Lecture 2}

\subsection{Thermodynamic Limit of the Pressure}

We first introduce some new notation. We denote
\begin{equation}
  \varepsilon_{\Lambda} := \{ \{ i, j \} \subset \Lambda : \underbrace{|| i - j || = 1}_{i \sim j ; \; i, j \; \text{n.n.}} \}.
\end{equation}

We regard $\varepsilon_{\Lambda}$ as the bulk with interactions across the boundary. Now, for a the bulk which \textit{does} have interactions across the boundary, we denote this as
\begin{equation}
  \varepsilon_{\Lambda}^{b} := \{ \{ i, j \} \subset \Z^d : i \sim j, \; i \in \Lambda \text{ or } j \in \Lambda \}.
\end{equation}

Now, let's consider the energy with \textit{empty boundary}: $\beta > 0$, and $h \in \R$. We calculate: 
\begin{align}
\label{eq:4}
  \sH_{\Lambda, \beta, h}^{\emptyset} : \Omega_{\Lambda} \rightarrow \R, \quad \sH_{\Lambda, \beta, h}(\omega) := - \beta \sum\limits_{\{ i, j \} \in \varepsilon_{\Lambda}} \omega_i \omega_j - h \sum\limits_{i \in \Lambda} \omega_i.
\end{align}

Moreover, we may write other thermodynamic quantities with the notion of the ``empty boundary condition'': 
\begin{align}
\label{eq:5}
  Z_{\Lambda, \beta, h}^{\emptyset} := \sum\limits_{\omega\in\Omega_{\Lambda}} e^{-\sH_{\Lambda, \beta, h}^{\emptyset}}, \quad \mu_{\Lambda, \beta, h}^{\emptyset}, \quad \left\langle \; , \;  \right\rangle_{\Lambda, \beta, h}^{\emptyset}, \quad \psi_{\Lambda}^{\emptyset}(\beta, h) := \frac{1}{\left| \Lambda \right|} \log Z_{\Lambda, \beta, h}^{\emptyset}.
\end{align}

Now, consider an infinite system where we have $\Omega := \{ +1, -1 \}^{\Z^{d}}$. We formalize the boundary conditions. Consider $n \in \Omega$, for example $\eta_i = \pm 1 \; \forall i \in \Z^d$. Then let
\begin{equation}
  \Omega_{\Lambda}^{\eta} := \{ \omega \in \Omega : \omega_i = \eta_i \; \forall i \in \Z^d \setminus \Lambda \}. 
\end{equation}

Then, the energy becomes:

\begin{align}
  \sH_{\Lambda, \beta, h} : \Omega &\rightarrow \R \\
  \omega &\mapsto \sH_{\Lambda, \beta, h}(\omega) := - \beta \sum\limits_{\{ i, j \} \in \varepsilon_{\Lambda}^{b}} \omega_i \omega_j - h \sum\limits_{i \in  \Lambda} \omega_i
\end{align}

Note that since $\omega \in \Omega_{\Lambda}^{\eta}$, then
\begin{equation}
  \sH_{\Lambda, \beta, h}(\omega) = \sH_{\Lambda, \beta, h}^{\emptyset}(\omega_{\Lambda}) - \beta \sum\limits_{\substack{\{ i, j \} : i \sim j \\ i \in \Lambda, j \in \Lambda^C}} \omega_i \eta_j.
\end{equation}
(Note that here, $\omega_j = \eta_j$.)

So then with $\omega_{\Lambda} = (\omega_i)_{i \in \Lambda}$, the partition function is 
\begin{align}
\label{eq:7}
  Z_{\lambda, \beta, h}^{\eta} &:= \sum\limits_{\omega \in \Omega_{\Lambda}^{\eta}} e^{-\sH_{\Lambda, \beta, h}(\omega)} \\
  \mu_{\Lambda, \beta, h}^{\eta} &, \; \psi_{\Lambda, \beta, h}(\beta, h).
\end{align}

\begin{Definition}{van Hove Convergence}
.  $(\Lambda_{n})_{n\in\N}, \; \Lambda_n \Subset \Z^d$ converges to $\Z^d$ in the sense of \index{van Hove}\textbf{van Hove} (or ``is a van Hove sequence'') if:
\begin{enumerate}
\item $\Lambda_n \uparrow \Z^d: \; \forall n , \Lambda_n \subset \Lambda_{n + 1}, \; \Z^d = \cup_{n \in \N} \Lambda_{n}$.
\item
  \begin{equation}
    \lim_{n \to \infty} \frac{\left| \partial^{\mathrm{in}} \Lambda_{\mathrm{in}} \right|}{\left| \Lambda_n \right|} \rightarrow 0
  \end{equation}
  where the inner boundary is
  \begin{equation}
    \partial^{\mathrm{in}}\Lambda := \{ i \in \Lambda : \exists j \in \Lambda^C \text{ such that } \lVert j - i \rVert = 1 \}
  \end{equation}
\end{enumerate}
  \textbf{\textit{Notation.}} We will denote convergence in the sense of van Hove by $\Lambda_n \Uparrow \Z^d$.
\end{Definition}

\begin{Example}{van Hove Sequences}
  .  First, let $\Lambda_{n} = \{ - n, \dots, n \}^{d}$, then
  \begin{align}
    \left| \partial^{\mathrm{in}} \Lambda_n \right| &= \mathcal{O}(n^{d-1}),\\
    \left| \Lambda_n \right| &= n^d.
  \end{align}
  This is a van Hove sequence. For a second example, now let $\Lambda_n = \{ - n, \dots, n \} \times \{ - n^2 , \dots, n^2 \}$ in $\Z^2$. Then 
  \begin{align}
  \label{eq:8}
    \left| \partial^{\mathrm{in}} \Lambda_n \right| &= \mathcal{O}(n^2), \\
    \left| \Lambda_n \right| &= cn^3.
  \end{align}
\end{Example}

\begin{Theorem}{van Hove sequence properties}[label=theo:vanHoveSeqProps]
.  
  \begin{enumerate}
  \item The limit $\psi(\beta, h) = \lim_{n \rightarrow \infty} \frac{1}{\left| \Lambda \right|} \log \Z^{\#}_{\Lambda_n, \beta, h} = \lim_{n \rightarrow \infty} \psi_{\Lambda_n}^{\#}(\beta, h)$ exists for all van Hove sequences, for every boundary condition $\# = \emptyset, \; \# = \eta \in \Omega$.

    The value does not depend on the precise choice of van Hove sequence.
    \item The value of the limit $\psi(\beta, h)$ does not depend on the precise choice of boundary condition, either.
    \item The map 
    \begin{align}
    \label{eq:9}
      (0, \infty) \times \R &\rightarrow \R \\
      (\beta, h) &\mapsto \psi(\beta, h)
    \end{align}
    is convex.
  \item $\forall \beta > 0, \; \psi(\beta, h) = \psi(\beta, -h)$.
  \end{enumerate}
\end{Theorem}

\begin{Definition}{Convex}[label=def:convex]
.  For a function to be \textbf{convex}\index{convex} means that $\forall (\beta_1, h_1), \; \forall (\beta_2, h_2), \; \forall t \in [0, 1]$, then
\begin{equation}
  \psi\Big( (1 - t) \beta_1 + t \beta_2, (1 - t) h_1 + t h_2\Big) \leq (1 - t) \psi(\beta_1, h_1) + t \psi(\beta_2, h_2).
\end{equation}
\end{Definition}

Proof of parts 1, 2 of~\ref{theo:vanHoveSeqProps} in Live Notes LN2.

\printindex

\end{document}
